\documentclass{article}
\usepackage{amsmath}
\usepackage{fullpage}
\usepackage{palatino}
\usepackage{pbox}
\usepackage[usenames,dvipsnames,svgnames,table]{xcolor}

\begin{document}

\newcommand{\path}{\mathbin{=}}
\newcommand{\trunc}[2]{\| #2 \|_{#1}}
\newcommand{\proj}[2]{| #2 |_{#1}}
\newcommand{\fpbox}[1]{\fbox{\pbox[t]{\textwidth}{#1}}}
\newcommand{\suc}{\mathtt{suc}}
\newcommand{\TODO}[1]{\textcolor{Olive}{\fbox{\textsc{todo}} {#1}}}

\section{Truncation and Connectedness}
\begin{itemize}
  \item[Contractible]

    \fpbox{
      $X$ is contractible.
      \\
      $X$ is a \emph{singleton}.
      \\
      $\sum_x\,\prod_y\,y \path_X x$.
    }

    Intuition: You pick a point $x$ and move $y$ around, and there is always
    a path connecting $x$ and $y$. The choice of the path should be continuous
    while $y$ is moving. Take the circle as an example:
    No continuous choices are possible when $y$ travels back to the base point
    along the loop, which is to say that the circle is not contractible.

  \item[Truncated]

    \fpbox{
      $X$ is $(-2)$-truncated.
      \\
      $X$ is contractible.
      \\
      $X$ is a singleton.
    }

    \fpbox{
      $X$ is $(-1)$-truncated.
      \\
      $X$ is a \emph{mere proposition}.
      \\
      $X$ is a subsingleton.
    }

    \fpbox{
      $X$ is $0$-truncated.
      \\
      $X$ is a set.
    }

    \fpbox{
      $X$ is $n$-truncated.
      \\
      $X$ is an $n$-type.
    }

    The $n$ here is the truncation level starting with $-2$.
    This predicate $T_n$ is defined inductively on $n$ as follows:
    \begin{align*}
      T_{-2}(X) &= \text{$X$ is contractible}
      \\
      T_{\suc(n)}(X) &= \prod_x \prod_y T_n(x \path_X y)
    \end{align*}
    ($T_n$ here is just some arbitrary symbol I chose.)

    The terminology ``\emph{mere} proposition'' is a compromise between two or more
    disciplines.  In type theory, propositions are types, but in many other areas,
    propositions do not have higher structures.

    \emph{Homotopy level} is the same except that the numbers start with $0$ (shifted by $2$).
    We never use it in our \textsc{Agda} library.

  \item[Truncation]

    \fpbox{
      $\trunc{n}{X}$
      \\
      $n$-truncation of $X$
    }

    \fpbox{
      $\proj{n}{x}$
      \\
      The projection from $X$ to $\trunc{n}{X}$
    }

    This is the projection of $X$ to the universe of $n$-types,
    satisfying the following universal property:

    \begin{quote}
      If $Y$ is an $n$-type and there is a map $f$ from $X$ to $Y$,
      and then there exists a (unique?)\footnote{\TODO{?}} map $\hat f$
      from $\trunc{n}{X}$ to $Y$ where $\hat f(\proj{n}{x}) = f(x)$.
    \end{quote}

    This is definable as an higher inductive type.

  \item[Connected]

    \fpbox{
      $X$ is \emph{path-connected}.
      \\
      $X$ is $0$-connected.
    }

    \fpbox{
      $X$ is \emph{simply-connected}.
      \\
      $X$ is $1$-connected.
    }

    \fpbox{
      $X$ is \emph{$n$-connected}.
      \\
      $\trunc{n}{X}$ is contractable.
    }

    People use words instead of numbers for common connectedness conditions.
\end{itemize}
\end{document}
