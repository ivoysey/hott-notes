\documentclass{article}
\usepackage{amsmath}
\usepackage{fullpage}
\usepackage{palatino}
\usepackage[usenames,dvipsnames,svgnames,table]{xcolor}
\title{Basics}
\author{Kuen-Bang Hou (Favonia)}

\begin{document}
\maketitle

\newcommand{\path}{\mathbin{=}}
\newcommand{\TODO}[1]{\textcolor{Olive}{\fbox{\textsc{todo}} {#1}}}

\section{Basic definitions}
\begin{itemize}
  \item
    $X$ is \emph{contractible}.\\
    $\Sigma (x{:}X)\,\Pi (y{:}X)\,x \path y$.
    
    Intuition: You pick a point $x$ and move $y$ around, and there is always
    a path connecting $x$ and $y$. The choice of the path should be continuous
    while $y$ is moving. Take the circle as an example:
    No continuous choices are possible when $y$ travels back to the base point
    along the loop, which is to say that the circle is not contractible.

  \item
    \TODO{Truncation levels. Homotopy levels.}

  \item
    \TODO{Truncation.}

  \item
    $X$ is \emph{path-connected}.
    \\
    $X$ is $0$-connected.
    
    $X$ is \emph{simply-connected}.
    \\
    $X$ is $1$-connected.
    
    $X$ is \emph{$n$-connected}.
    \\
    $\| X \|_n$ is contractable.

    People use words instead of numbers for common connectedness conditions.
\end{itemize}
\end{document}
