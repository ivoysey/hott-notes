\documentclass{article}
\usepackage{amsmath}
\usepackage{fullpage}
\usepackage{palatino}
\usepackage{pbox}
\usepackage{fancyvrb}
\usepackage{mdframed}
\usepackage[usenames,dvipsnames,svgnames,table]{xcolor}

\begin{document}

\newcommand{\path}{\mathbin{=}}
\newcommand{\trunc}[2]{\| #2 \|_{#1}}
\newcommand{\proj}[2]{| #2 |_{#1}}
\newcommand{\fpbox}[1]{\fbox{\pbox[t]{\textwidth}{#1}}}
\newcommand{\suc}{\mathtt{suc}}
\newcommand{\TODO}[1]{\textcolor{Olive}{\fbox{\textsc{todo}} {#1}}}

\begin{mdframed}
  \paragraph{\textsc{Disclaimer}}
  Due to my own background, the presentation here is probably biased heavily toward type theory.
  I am still learning the classical algebraic topology, and these are just study notes that
  help me put everything together in my mind.
\end{mdframed}

\section{What is Homotopy Type Theory?}

Homotopy type theory is some type theory%
\footnote{A good understanding of the computational behaviors
  is lacking and thus the ``type theory'' is not fully working (yet).}
exploring the possibility of
giving inductive definitions to many homotopy invariants in (algebraic) topology.
The current formulation is more or less the intentional type theory
with two additional bits:
\begin{enumerate}
  \item Higher inductive types

    An extension to inductive types which makes possible
    a wide range of interesting spaces.

  \item The univalence axiom

    Intuitively, this asserts that isomorphic (homotopy equivalent) types are the same,
    which effectively injects a path to the univalence
    for each isomorphism between two types.
    As a result, UIP (uniqueness of identity types) in general
    does not hold since there are possibly more than one isomorphism
    between two types.
\end{enumerate}

From the type-theoretical point of view,
the current formulation with additional axioms is not ideal;
the additional axioms very likely break the harmonic
formation-introduction-elimination-computation style,
which originates from important philosophical principles behind type theory.
In the case of the univalent axiom,
it is an additional introduction rule
that does not have (obvious) matching elimination and computation rules.
We believe that one day we will restore the harmony again,
integrating the higher inductive types and the univalent axiom
without violating the philosophy behind type theory.

Basic concepts in type theory and homotopy theory match beautifully.
Paths and homotopies are identity types (without the UIP),
and homotopies between two paths are simply identity types over identity types.
A type family $P$ indexed by $A$ is a fibration from the total space $\sum_x P(x)$ to $A$,
where $P(x)$ is the fiber over $x$.
A function in type theory is a functor between spaces,
mapping not only points but also all the paths and higher structures.
Substitution is transporting of a point in the codomain.

One of the easiest example of higher inductive types is the circle.
If we want to reason about the circle in an abstract way
(rather than the analytic formulation),
it can be a base point and a loop based at that point.
Formally speaking, the circle can be expressed in the following way:
\newcommand{\circletype}{\ensuremath{S^1}}
\begin{Verbatim}[commandchars=\\\{\}]
  data \circletype where
    base : \circletype
    loop : base = base
\end{Verbatim}
where the \texttt{loop} constructor inserts a generator for paths between
two points of the same type. Traditional inductive types
only allow generators for points.

Now let's move on to the elimination and computation rules.
To make a function $f$ of type $\circletype \to X$ for some type $X$,
one needs to provide a point $x$ in $X$ and a loop based at $x$.
This is similar to the induction principle for other types:
All generators must be given with matching ingredients.
In this case $x$ matches \texttt{base} and the loop based at $x$
matches \texttt{loop}, forming a proper functor from $\circletype$ to $X$.

One fundamental difference between type-theoretical proofs
and those in the classical homotopy theory is that, the former requests
all coherent data right in the induction principle, while the latter
usually deals with the function for points and the coherence data separately.
For example, we need to show that the constructor \texttt{loop}
is mapped before even having the function,
while in classical theory you define a set-theoretical mapping
before demonstrating that \texttt{loop} is also mapped.

\section{Truncation and Connectedness}
\begin{itemize}
  \item[Contractible]

    \fpbox{
      $X$ is contractible.
      \\
      $X$ is a \emph{singleton}.
      \\
      $\sum_x\,\prod_y\,y \path_X x$.
    }

    Intuition: You pick a point $x$ and move $y$ around, and there is always
    a path connecting $x$ and $y$. The choice of the path should be continuous
    while $y$ is moving. Take the circle as an example:
    No continuous choices are possible when $y$ travels back to the base point
    along the loop, which is to say that the circle is not contractible.

  \item[Truncated]

    \fpbox{
      $X$ is $(-2)$-truncated.
      \\
      $X$ is contractible.
      \\
      $X$ is a singleton.
    }

    \fpbox{
      $X$ is $(-1)$-truncated.
      \\
      $X$ is a \emph{mere proposition}.
      \\
      $X$ is a subsingleton.
    }

    \fpbox{
      $X$ is $0$-truncated.
      \\
      $X$ is a set.
    }

    \fpbox{
      $X$ is $n$-truncated.
      \\
      $X$ is an $n$-type.
    }

    The $n$ here is the truncation level starting with $-2$.
    This predicate is defined inductively on $n$ as follows:
    \begin{align*}
      \text{$X$ is $(-2)$-truncated} &= \text{$X$ is contractible}
      \\
      \text{$X$ is $\suc(n)$-truncated} &= \prod_x\,\prod_y\,\text{$(x \path_X y)$ is $n$-truncated}
    \end{align*}

    The terminology ``\emph{mere} proposition'' is a compromise between lots of
    disciplines.  In type theory, propositions are as types, but in many other areas,
    propositions are $(-1)$-truncated.

    \emph{Homotopy level} is the same except that the numbers start with $0$ (shifted by $2$).
    We never use it in our \textsc{Agda} library.

  \item[Truncation]

    \fpbox{
      $\trunc{n}{X}$
      \\
      $n$-truncation of $X$
    }

    \fpbox{
      $\proj{n}{x}$
      \\
      The projection from $X$ to $\trunc{n}{X}$
    }

    This is the projection of $X$ to the universe of $n$-types,
    satisfying the following universal property:

    \begin{quote}
      If $Y$ is an $n$-type and there is a map $f$ from $X$ to $Y$,
      and then there exists a (unique?)\footnote{\TODO{?}} map $\hat f$
      from $\trunc{n}{X}$ to $Y$ where $\hat f(\proj{n}{x}) = f(x)$.
    \end{quote}

    This is definable as an higher inductive type.

  \item[Connected]

    \fpbox{
      $X$ is \emph{path-connected}.
      \\
      $X$ is $0$-connected.
    }

    \fpbox{
      $X$ is \emph{simply-connected}.
      \\
      $X$ is $1$-connected.
    }

    \fpbox{
      $X$ is \emph{$n$-connected}.
      \\
      $\trunc{n}{X}$ is contractable.
    }

    People use words instead of numbers for common connectedness conditions.
\end{itemize}
\end{document}
